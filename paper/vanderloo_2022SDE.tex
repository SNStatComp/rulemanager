% LaTeX template for the UNECE Worksession on Statistical Data Editing 2015, Budapest
% Last changed: 2015-04-10
% Author: Matthias Templ (using an earlier version by Steven Vale and Daniel Kilchmann)
% Revision and modification by Frederic Picard, Statistics Canada
% April 10 2015
% Revision and modification by Daniel Kilchmann, Swiss Federal Statistical Office
% December 7 2016, April 5 2018, October 8 2019, May 18 2022,

% DISCLAIMER: The UNECE document class style and the LaTeX-template
% are provided without any warranty at all and without any commitment
% to maintain or correct the code. In no event are UNECE or the
% author liable for any damages arising from the use of it.

% The uneceart document class style must be used for right numbering etc.
\documentclass[a4paper,11pt]{style/uneceart}

\usepackage{style/UNECE2022}
\usepackage{enumerate}
\usepackage{xcolor}
\definecolor{unece_color}{RGB}{84, 141, 212}

%%% ----------------------------------------------------------------------
%%% ---   please fill-in your personal data:   ---------------------------

%% the title of your contribution in capital letters:
\newcommand{\TITLE}{\textbf{Validation Rule Management} \\}
%% The type of your paper (contributed or invited)
\newcommand{\TYPE}{\textbf{Invited paper}\vspace*{5mm}\\}

%% Topic (i,ii,iii,iv,v,..) 
%\newcommand{\TOPIC}{Theme (i):  \\  }
%\newcommand{\TOPIC}{Theme (ii):\\  }
%\newcommand{\TOPIC}{Theme (iii): ):  \\  }
%\newcommand{\TOPIC}{Theme (iv):  \\  }
%\newcommand{\TOPIC}{Theme (v):  \\  }
%\newcommand{\TOPIC}{Theme (XXX): XXXXXXX(choose on the themes from the ones in the Latex comments above) \\  }

%% No. of the working paper (fill-in as soon you know it)
\newcommand{\WP}{WP. \\   }
%% author:
\newcommand{\AUTHOR}{Mark van der Loo, Edwin de Jonge, Olav ten Bosch }
%% your organisation
\newcommand{\ORGANISATION}{Author(s) (Statistics Netherlands, The Netherlands}
\newcommand{\EMAIL}{mpj.vanderloo@cbs.nl}




\begin{document}

\input{UNECE2022cover.tex}





% Introduction
\section{\hspace*{1ex}INTRODUCTION}  %% The extra hspace is only to align the numbering more or less

\paragraph foo bar

\cite{loo2021data}

% First section
\section{User stories}  \label{sc:A}

As a statistics producer, I want to
\begin{enumerate}
\item Create, Update, and Delete rules so I can fix my current understanding
   of a statistical domain in the form of a formal ruleset.
\item Select a set of rules so I can apply them to my data.
\item Determine the order of rule execution so I have full control over
   data processing and validation.
\item Be able to trace the evolution of my rules and rule sets so I can (a) give full account
   of my production runs, and (b) reproduce production runs.
\item Temporarily remove a rule from one or more rule sets so I can handle exceptional and transient
   data circumstances. This temporary removal should be documented.
\item Temporarily update a rule from one or more rulesets so I can handle exceptional and transient
   data circumstances. This temoporary update should be documented.
\end{enumerate}

As a statistical organization, I want to
\begin{enumerate}
\item Promote reuse of rules.
\item Promote transparency and learning accross production systems, by
      comparing and benchmarking rules and rule sets.
\end{enumerate}



\subsection{Formal considerations}
\paragraph The most important thing to do in developing any software system is
designing the formal concepts and their relations, that are necessary to
support its function. Conceptually we work with two data structures: a rule
repository and a rule sequence. We next define basic operations that enable us
to represent the above user stories. 

\paragraph A \emph{rule repository} $R=\{r_1,\ldots,r_m\}$ is a finite set of
rules.  The contents of a rule repository may be time-dependent as new rules
can be added and obsolete rules can be removed. The purpose of a rule
repository is to offer a rich set of rules from which selections can be made
depending on the purpose. For example, data processing rules may differ between
running a flash estimate and an updated estimate, although strong overlap is
likely to exist.

\paragraph A \emph{rule sequence} $L = (r_1,\ldots,r_n)$ is a finite ordered
sequence of rules, where rules come from a predefined, unordered repository.
Rule sequences are applied to data in order, so by function composition we may
also think of $L(t)$ as a data transformation function defined by the
composition $r_n\circ\cdots\circ r_1$. Creating a rule sequence consists of
selecting a subset from a rule repository and placing them in a certain order.
The number of ways this can be done grows quickly with the number of rules. For
a repository of cardinality $m$, there are principally $\sum_{k=0}^m {m\choose
k}!$ such lists, assuming that each rule can occur only once in a rule
sequence. Although in practice we expect that rule sequences will have
significant permutational symmetry in the sense that rule order can be
(partially) changed in many cases without affecting the action of the sequence
on data sets.

\paragraph From the user point of view, a rule sequence is the main conceptual
structure to work with. Each production run is controlled by one or more of
such lists, that may be updated as data circumstances evolve over time. From
the organizational point of view, the rule repository is more interesting. The
repository $R(t)$ offers the possibility to avoid duplicates and to discover
common patterns accross rules and rule evolution. In the case of transformation
rules, it offers the possibility to see whether different choices are made in
similar data circumstances; in the case of data validation rules it offers the
possibility to see whether different quality restrictions are defined for the
same variable. The collection of rule sequences derived from the rule
repository are also interesting from the point of view of the organisation as
it allows for benchmarking and comparing production systems.

\paragraph The basic operations on rule repositories are addition and removal
of a single rule. These can be represented as set union $\cup$ and set
difference $-$. For example, if $R=\{r_1,r_2,r_3\}$ then $R\cup
\{r_4\}=\{r_1,r_2,r_3,r_4\}$ and $R-\{r_2\}=\{r_1,r_3\}$.

\paragraph The basic operations on rule sequences are permutation,
concatenation of a single rule, and removal of the last rule. A
permutation $\sigma_{ij}$ swaps the position of the ith and jth
element: 
$$
\sigma_{ij}(r_1,\ldots, r_i,\ldots,r_j,\ldots, r_n)=
            (r_1,\ldots, r_j,\ldots,r_i,\ldots, r_n).
$$
Concatenation $\oplus$ adds an element from a rule repository at the end of a
list:
$$
(r_1,\ldots,r_n) \oplus (r) = (r_1,\ldots,r_n,r),\: r\in R.
$$
Deletion $\ominus$ is a unary operator that removes the last element of a list.
An empty list is left intact. In notation:
$$
\ominus (r_1,\ldots,r_{n-1},r_n) = (r_1,\ldots,r_{n-1}),\textrm{ and } \ominus()=().
$$
Together these operations allow for deleting or inserting any rule from and
into any position. For example, in order to remove the jth element from a rule
sequence, we just permute with consequtive neighbours until it is the last
element and apply the $\ominus$ operator. Conversely, we can insert a rule at
any position by concatenating it at the end with $\oplus$ and then use a
sequence of permutations to move it into the desired position.



\subsection{Implementation design} \label{sc:B}



\section{The rulemanager package}



\bibliographystyle{plainnat}
\bibliography{vanderloo}



\end{document}
